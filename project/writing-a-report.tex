\documentclass[a4paper,11pt]{article}

\usepackage{graphicx}  %%% for including graphics
\usepackage{url}       %%% for including URLs
\usepackage{palatino,eulervm}
%\usepackage{times}
\usepackage{natbib}
\usepackage[T1]{fontenc}
\usepackage[utf8]{inputenc}
\usepackage[margin=25mm]{geometry}
\usepackage{hyperref}
\usepackage{sectsty}

% Redefine the section headings
\sectionfont{\normalsize \bfseries}
\subsectionfont{\normalsize \normalfont \itshape}
\subsubsectionfont{\normalsize \normalfont}
%\paragraphfont{\normalfont \MakeUppercase} %itshape
\paragraphfont{\normalfont \itshape} %itshape



\title{\textbf{How to write a project report}}
\date{\normalsize \today}

\author{\normalsize Simon Dobnik\\
       \normalsize Centre for Linguistic Theory and Studies in Probability (CLASP) \\
       \normalsize University of Gothenburg \\
       \normalsize \texttt{simon.dobnik@gu.se}
  % \and Someone Else\\
  %      Another Affiliation\\
  %      \texttt{another@email.org}
}


\begin{document}
\maketitle
% \thispagestyle{empty}
% \pagestyle{empty}

\section{About a project report}\label{how-to-write-a-project-report}

Write the project report like a scientific article. A report should be a
self contained piece of good scientific writing which means that it
should be clear to any reader in language technology. Although this is
strictly not necessary it also good to format it, like a scientific
article. In computational linguistics or language technology the
\href{http://acl2017.org/calls/papers/}{ACL style} is very popular
(\href{http://acl2017.org/downloads/acl17-latex.zip}{LaTeX} and
\href{http://acl2017.org/downloads/acl17-word.zip}{Word}).


\section{Headings to include}

\subsection{Abstract}\label{abstract}

Write a few sentences which summarise the work in a way that is
understandable to someone working in language technology. Why? How?
Results. Conclusions. Not more than 100 words.

\subsection{Introduction}\label{introduction}

Give some background about the problem you are trying to solve. You do
this by gradually focusing into to question you are going to investigate
by discussion of the previous work ending with the work that this paper
more closely relates to. What question have been answered in this work
and which questions are outstanding that we will be dealt with in this
paper? Finally, state a list of steps that will be taken to address
these issues and a description of how this paper is organised (In
Section 2 we\ldots{})

\subsection{Materials and methods}\label{materials-and-methods}

Here you describe your toolkit and tools that you will use to test and
answer these questions. Describe how the experiment(s) have been carried
out in detail. What is the hypothesis that the experiment should test or
more generally what should the experiment show? Not that hypotheses are
different and more specific than open research questions from the
introduction. There are a way of testing these research questions.

\subsection{Results}\label{results}

What did you find out? First show the data and then draw conclusions of
the data to support your previous hypotheses/predictions. Have
hypotheses been confirmed or rejected? The conclusions are your results.

Support your argument with figures and tables. It should be possible to
read figures and tables without the text and understand the text without
looking at figures and tables. Refer to each figure or table at least
once in the text.

\subsection{Discussion}\label{discussion}

In this section you discuss your results in relation to the open
research questions from the introduction. To what extent do result
answer them? If applicable, look into the literature for further
explanations for your findings which may give you further suggestions:
new findings that could not be anticipated from the beginning. Emphasise
and discuss in what ways your work is relevant for the chosen research
area.

\subsection{Conclusions and further
work}\label{conclusions-and-further-work}

Summarise what has been done in the preceding sections and point out
areas where the work described in this report could be extended in the
future.


\section{Reports about joint work}

If a report is about joint work of several people, then you should
mark unambiguously (as with the code, datasets and other resources)
who wrote what part of the report. For example, this can be done by
putting your initials in the headings above the text that you wrote;
e.g. \textbf{Pre-training features with CNN (FT)} where FT are your
initials. Try to use headings that describe the sub-parts of the
project that you were responsible for, otherwise indicate the relevant
parts of the text with your initials inline.

As always, you may and you are encouraged to discuss the projects with
each other (and with the teachers of course!) but what you hand in
must be written (or programmed) by you rather than being copied from
someone else. In case of doubt, please have a look at
\href{http://www.cl.cam.ac.uk/teaching/exams/plagiarism.html}{https://goo.gl/JbhCVU}
or ask us.


% \bibliographystyle{chicago}
% \bibliography{bibliography}

\end{document}

%%% Local Variables:
%%% mode: latex
%%% mode: flyspell
%%% TeX-master: t
%%% TeX-engine: xetex
%%% TeX-PDF-mode: t
%%% coding: utf-8
%%% ispell-local-dictionary: "british"
%%% End:
